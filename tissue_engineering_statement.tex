\documentclass[12pt,a4paper,twocolumn]{article}
\usepackage[margin=2cm]{geometry}
\usepackage[compact]{titlesec}

\renewcommand{\familydefault}{\sfdefault}

\linespread{0.96}

\begin{document}

\title{\vspace{-3em}Tissue Engineering\vspace{-3em}}
\date{}
\maketitle

\thispagestyle{empty}

\section{Introduction}
Tissue engineering can be considered to be causing tissue to grow in ways outside of normal growth and repair. This ranges from causing tissues to heal that would otherwise not, such as the use of stem cells to repair damage to the spinal cord, through to complete in-vitro growth of tissue. The current driving force behind this research is regenerative medicine, which of course is an extremely important and promising field in its own right. However, what makes me particularly excited by this area are possible applications further downstream. These can be broken into scientific and industrial results.

\section{Scientific}
There is already considerable excitement about recent advances in tissue engineering being able to provide cheap tissue for experiments, free of ethical constraints, which would lower the bar to entry. However, tissue engineering also has the potential to vastly improve the quality of investigations, as well as the quantity. 

Tissue that is maintained in-vitro, regardless of its initial origin, is isolated from the rest of the organism. This removes the feedback loops, which while biologically vital, make a reductionist, cause and effect understanding of functionality difficult to establish. With an isolated tissue we can know that the response to any perturbation is not modulated by another, unknown part of the system, whether that other part is attempting to compensate, or being otherwise affected in a way that has results that could be confused with the results from the target tissue.

Furthermore, isolation of tissue implies complete control and knowledge over the inputs and outputs from the tissue. This added precision in control of environment allows for a more subtle understanding of the exact effects of chemical environment, and also helps to rule out confounding effects, such as stress levels in live animals.

Finally, if one increases consistency of experimental materials, and reduces costs, precision is sure to rise. In particular, in-vitro tissue would be amenable to higher levels of automation in processing than whole plants or animals, further pushing down cost and increasing consistency of treatment.

\section{Industrial}
It is possible to imagine almost limitless industrial possibilities for tissue engineering, but if we limit ourselves to only the near term, there are still many exciting prospects.

In-vitro meat is one of the most obvious possibilities from tissue engineering. This could allow vegetable matter to be more efficiently converted to meat, increasing total food availability, and decreasing price, while also reducing the huge environmental impact of farming, in particular of cattle. There are obviously huge ethical benefits, but also finer control of production would allow for healthier meat to be produced, helping to fight obesity and associated diseases.

At present, one of the major challenges to the growth of tissue in-vitro is higher level structure, particularly the intracellular matrix and blood vessels. While this is obviously a challenge for regenerative medicine applications, it does provide a funding boon to novel micro-scale manufacturing techniques, such as some 3D printing techniques. These techniques are being experimented with to construct matrices from biodegradable materials such as PLA, which cells can then grow over, and finally degrade. Hopefully these plastic matrices will gradually become unnecessary as we understand how to initiate natural creation of these structures, but there are nevertheless many applications for the associated technology, particularly in the construction of 3D components in electronic devices constructed using 2D stereolithography.

This ability to construct micro-scale, biocompatable 3D structures could have applications in the construction of biological sensors. In medicine this could be combined with microfluidics to allow novel diagnostic techniques using only very small sample sizes, perhaps continually sampled over long periods.

\section{Conclusion}
The industrial and scientific impact from tissue engineering may extend far beyond regenerative medicine, though this will only happen if our understanding of the underlying control mechanisms is sound. We may yet find processes which are necessary, but which are not well understood enough to be replicated in-vitro. However, if tissue engineering does live up to its promise, then its effects will be far-reaching and profound.

\end{document}
